
% This LaTeX was auto-generated from an M-file by MATLAB.
% To make changes, update the M-file and republish this document.



    
    
      \subsection{lossSat.m}

\begin{par}
\textbf{Summary:} Compute expectation and variance of a saturating cost $1 - \exp(-(x-z)^T*W*(x-z)/2)$ and their derivatives, where x \ensuremath{\tilde{\;}} N(m,S), z is a (target state), and W is a weighting matrix
\end{par} \vspace{1em}

\begin{verbatim}function [L, dLdm, dLds, S, dSdm, dSds, C, dCdm, dCds] = lossSat(cost, m, s)\end{verbatim}
    \begin{par}
\textbf{Input arguments:}
\end{par} \vspace{1em}

\begin{verbatim}  cost
   .z:     target state                                               [D x 1]
   .W:     weight matrix                                              [D x D]
 m         mean of input distribution                                 [D x 1]
 s         covariance matrix of input distribution                    [D x D]\end{verbatim}
    \begin{par}
\textbf{Output arguments:}
\end{par} \vspace{1em}
\begin{verbatim}L               expected loss                                  [1   x    1 ]
dLdm            derivative of L wrt input mean                 [1   x    D ]
dLds            derivative of L wrt input covariance           [1   x   D^2]
S               variance of loss                               [1   x    1 ]
dSdm            derivative of S wrt input mean                 [1   x    D ]
dSds            derivative of S wrt input covariance           [1   x   D^2]
C               inv(S) times input-output covariance           [D   x    1 ]
dCdm            derivative of C wrt input mean                 [D   x    D ]
dCds            derivative of C wrt input covariance           [D   x   D^2]\end{verbatim}
\begin{par}
Copyright (C) 2008-2013 by Marc Deisenroth, Andrew McHutchon, Joe Hall, and Carl Edward Rasmussen.
\end{par} \vspace{1em}
\begin{par}
Last modified: 2013-05-28
\end{par} \vspace{1em}


\subsection*{High-Level Steps} 

\begin{enumerate}
\setlength{\itemsep}{-1ex}
   \item Expected cost
   \item Variance of cost
   \item inv(s)*cov(x,L)
\end{enumerate}

\begin{lstlisting}
function [L, dLdm, dLds, S, dSdm, dSds, C, dCdm, dCds] = lossSat(cost, m, s)
\end{lstlisting}


\subsection*{Code} 


\begin{lstlisting}
% some precomputations
D = length(m); % get state dimension

% set some defaults if necessary
if isfield(cost,'W'); W = cost.W; else W = eye(D); end
if isfield(cost,'z'); z = cost.z; else z = zeros(D,1); end

SW = s*W;
iSpW = W/(eye(D)+SW);

% 1. Expected cost
L = -exp(-(m-z)'*iSpW*(m-z)/2)/sqrt(det(eye(D)+SW)); % in interval [-1,0]

% 1a. derivatives of expected cost
if nargout > 1
  dLdm = -L*(m-z)'*iSpW;  % wrt input mean
  dLds = L*(iSpW*(m-z)*(m-z)'-eye(D))*iSpW/2;  % wrt input covariance matrix
end

% 2. Variance of cost
if nargout > 3
  i2SpW = W/(eye(D)+2*SW);
  r2 = exp(-(m-z)'*i2SpW*(m-z))/sqrt(det(eye(D)+2*SW));
  S = r2 - L^2;
  if S < 1e-12; S=0; end % for numerical reasons
end

% 2a. derivatives of variance of cost
if nargout > 4
  % wrt input mean
  dSdm = -2*r2*(m-z)'*i2SpW-2*L*dLdm;
  % wrt input covariance matrix
  dSds = r2*(2*i2SpW*(m-z)*(m-z)'-eye(D))*i2SpW-2*L*dLds;
end

% 3. inv(s)*cov(x,L)
if nargout > 6
    t = W*z - iSpW*(SW*z+m);
    C = L*t;
    dCdm = t*dLdm - L*iSpW;
    dCds = -L*(bsxfun(@times,iSpW,permute(t,[3,2,1])) + ...
                                    bsxfun(@times,permute(iSpW,[1,3,2]),t'))/2;
    dCds = bsxfun(@times,t,dLds(:)') + reshape(dCds,D,D^2);
end

L = 1+L; % bring cost to the interval [0,1]
\end{lstlisting}
