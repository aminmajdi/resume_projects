
% This LaTeX was auto-generated from an M-file by MATLAB.
% To make changes, update the M-file and republish this document.



    
    
      \subsection{gp0d.m}

\begin{par}
\textbf{Summary:} Compute joint predictions and derivatives for multiple GPs with uncertain inputs. Predictive variances contain uncertainty about the function, but no noise. If gpmodel.nigp exists, individial noise contributions are added.
\end{par} \vspace{1em}
\begin{verbatim}function [M, S, V, dMdm, dSdm, dVdm, dMds, dSds, dVds] = gp0d(gpmodel, m, s)\end{verbatim}
\begin{par}
\textbf{Input arguments:}
\end{par} \vspace{1em}
\begin{verbatim}gpmodel    GP model struct
  hyp      log-hyper-parameters                                  [D+2 x  E ]
  inputs   training inputs                                       [ n  x  D ]
  targets  training targets                                      [ n  x  E ]
  nigp     (optional) individual noise variance terms            [ n  x  E ]
m          mean of the test distribution                         [ D  x  1 ]
s          covariance matrix of the test distribution            [ D  x  D ]\end{verbatim}
\begin{par}
\textbf{Output arguments:}
\end{par} \vspace{1em}
\begin{verbatim}M          mean of pred. distribution                            [ E  x  1 ]
S          covariance of the pred. distribution                  [ E  x  E ]
V          inv(s) times covariance between input and output      [ D  x  E ]
dMdm       output mean by input mean                             [ E  x  D ]
dSdm       output covariance by input mean                       [E*E x  D ]
dVdm       inv(s)*input-output covariance by input mean          [D*E x  D ]
dMds       ouput mean by input covariance                        [ E  x D*D]
dSds       output covariance by input covariance                 [E*E x D*D]
dVds       inv(s)*input-output covariance by input covariance    [D*E x D*D]\end{verbatim}
\begin{par}
Copyright (C) 2008-2013 by Marc Deisenroth, Andrew McHutchon, Joe Hall, and Carl Edward Rasmussen.
\end{par} \vspace{1em}
\begin{par}
Last modified: 2013-05-24
\end{par} \vspace{1em}


\subsection*{High-Level Steps} 

\begin{enumerate}
\setlength{\itemsep}{-1ex}
   \item If necessary, compute kernel matrix and cache it
   \item Compute predicted mean and inv(s) times input-output covariance
   \item Compute predictive covariance matrix, non-central moments
   \item Centralize moments
   \item Vectorize derivatives
\end{enumerate}

\begin{lstlisting}
function [M, S, V, dMdm, dSdm, dVdm, dMds, dSds, dVds] = gp0d(gpmodel, m, s)
\end{lstlisting}


\subsection*{Code} 


\begin{lstlisting}
% If no derivatives required, call gp0
if nargout < 4; [M S V] = gp0(gpmodel, m, s); return; end

persistent K iK beta oldX oldn;
[n, D] = size(gpmodel.inputs);    % number of examples and dimension of inputs
E = size(gpmodel.targets,2);                               % number of outputs
X = gpmodel.hyp;                              % short hand for hyperparameters

% 1) if necessary: re-compute cached variables
if numel(X) ~= numel(oldX) || isempty(iK) || sum(any(X ~= oldX)) || n ~= oldn
  oldX = X; oldn = n;
  iK = zeros(n,n,E);  K = iK; beta = zeros(n,E);

  for i=1:E                                              % compute K and inv(K)
    inp = bsxfun(@rdivide,gpmodel.inputs,exp(X(1:D,i)'));
    K(:,:,i) = exp(2*X(D+1,i)-maha(inp,inp)/2);
    if isfield(gpmodel,'nigp')
      L = chol(K(:,:,i) + exp(2*X(D+2,i))*eye(n) + diag(gpmodel.nigp(:,i)))';
    else
      L = chol(K(:,:,i) + exp(2*X(D+2,i))*eye(n))';
    end
    iK(:,:,i) = L'\(L\eye(n));
    beta(:,i) = L'\(L\gpmodel.targets(:,i));
  end
end

k = zeros(n,E); M = zeros(E,1); V = zeros(D,E); S = zeros(E);      % initialize
dMds = zeros(E,D,D); dSdm = zeros(E,E,D);
dSds = zeros(E,E,D,D); dVds = zeros(D,E,D,D); T = zeros(D);

inp = bsxfun(@minus,gpmodel.inputs,m');                     % centralize inputs

% 2) compute predicted mean and inv(s) times input-output covariance
for i=1:E
  iL = diag(exp(-X(1:D,i))); % inverse length scales
  in = inp*iL;
  B = iL*s*iL+eye(D); LiBL = iL/B*iL;
  t = in/B;
  l = exp(-sum(in.*t,2)/2); lb = l.*beta(:,i);
  tL = t*iL;
  tlb = bsxfun(@times,tL,lb);
  c = exp(2*X(D+1,i))/sqrt(det(B));
  M(i) = c*sum(lb);
  V(:,i) = tL'*lb*c;                     % inv(s) times input-output covariance
  dMds(i,:,:) = c*tL'*tlb/2 - LiBL*M(i)/2;
  for d = 1:D
    dVds(d,i,:,:) = c*bsxfun(@times,tL,tL(:,d))'*tlb/2 - LiBL*V(d,i)/2 ...
      - (V(:,i)*LiBL(d,:) + LiBL(:,d)*V(:,i)')/2;
  end
  k(:,i) = 2*X(D+1,i)-sum(in.*in,2)/2;
end
dMdm = V'; dVdm = 2*permute(dMds,[2 1 3]);                  % derivatives wrt m

iell2 = exp(-2*gpmodel.hyp(1:D,:));
inpiell2 = bsxfun(@times,inp,permute(iell2,[3,1,2])); % N-by-D-by-E

% 3) compute predictive covariance matrix, non-central moments
for i=1:E
  ii = inpiell2(:,:,i);

  for j=1:i
    R = s*diag(iell2(:,i)+iell2(:,j))+eye(D);
    t = 1/sqrt(det(R));
    ij = inpiell2(:,:,j);
    L = exp(bsxfun(@plus,k(:,i),k(:,j)')+maha(ii,-ij,R\s/2));
    if i==j
      iKL = iK(:,:,i).*L;
      s1iKL = sum(iKL,1);
      s2iKL = sum(iKL,2);
      S(i,j) = t*(beta(:,i)'*L*beta(:,i) - sum(s1iKL));
      zi = ii/R;
      bibLi = L'*beta(:,i).*beta(:,i); cbLi = L'*bsxfun(@times, beta(:,i), zi);
      r = (bibLi'*zi*2 - (s2iKL' + s1iKL)*zi)*t;
      for d = 1:D
        T(d,1:d) = 2*(zi(:,1:d)'*(zi(:,d).*bibLi) + ...
          cbLi(:,1:d)'*(zi(:,d).*beta(:,i)) - zi(:,1:d)'*(zi(:,d).*s2iKL) ...
          - zi(:,1:d)'*(iKL*zi(:,d)));
        T(1:d,d) = T(d,1:d)';
      end
    else
      zi = ii/R; zj = ij/R;
      S(i,j) = beta(:,i)'*L*beta(:,j)*t;
      S(j,i) = S(i,j);

      bibLj = L*beta(:,j).*beta(:,i);
      bjbLi = L'*beta(:,i).*beta(:,j);
      cbLi = L'*bsxfun(@times, beta(:,i), zi);
      cbLj = L*bsxfun(@times, beta(:,j), zj);

      r = (bibLj'*zi+bjbLi'*zj)*t;
      for d = 1:D
        T(d,1:d) = zi(:,1:d)'*(zi(:,d).*bibLj) + ...
          cbLi(:,1:d)'*(zj(:,d).*beta(:,j)) + zj(:,1:d)'*(zj(:,d).*bjbLi) + ...
          cbLj(:,1:d)'*(zi(:,d).*beta(:,i));
        T(1:d,d) = T(d,1:d)';
      end
    end

    dSdm(i,j,:) = r - M(i)*dMdm(j,:)-M(j)*dMdm(i,:);
    dSdm(j,i,:) = dSdm(i,j,:);
    T = (t*T-S(i,j)*diag(iell2(:,i)+iell2(:,j))/R)/2;
    T = T - reshape(M(i)*dMds(j,:,:) + M(j)*dMds(i,:,:),D,D);
    dSds(i,j,:,:) = T;
    dSds(j,i,:,:) = T;
  end

  S(i,i) = S(i,i) + exp(2*X(D+1,i));
end
% 4) centralize moments
S = S - M*M';

% 5) vectorize derivatives
dMds = reshape(dMds,[E D*D]);
dSds = reshape(dSds,[E*E D*D]); dSdm = reshape(dSdm,[E*E D]);
dVds = reshape(dVds,[D*E D*D]); dVdm = reshape(dVdm,[D*E D]);
\end{lstlisting}
