\documentclass{article}
\renewcommand{\rmdefault}{psbx}
\usepackage[utf8]{inputenc}
\usepackage[T1]{fontenc}
\usepackage{textcomp}
\usepackage{eulervm}
\usepackage{amsmath}
\usepackage{amssymb}

\setlength{\textwidth}{160mm}
\setlength{\oddsidemargin}{0mm}
\setlength{\parindent}{0 mm}

\newcommand{\bfm}{{\vec m}}
\newcommand{\bfs}{{\mat s}}
\newcommand{\bfz}{{\vec z}}
\newcommand{\E}{{\mathbb E}}
\newcommand{\V}{{\mathbb V}}
\newcommand{\T}{^{\top}}
\renewcommand{\vec}[1]{\boldsymbol{#1}}
\newcommand{\mat}[1]{\boldsymbol{#1}}
\newcommand{\inv}{^{-1}}

\title{Robotic Unicycle Loss Function}
\author{Marc Peter Deisenroth, Philipp Hennig and Carl Edward Rasmussen}
\date{October 31st, 2011}

\begin{document}

\maketitle
The state of the system is given as
\begin{equation*}
\vec x = 
\begin{bmatrix}
\dot\theta & \dot\phi &\dot\psi_w & \dot\psi_f & \dot\psi_t & \theta &
\phi & \psi_w & \psi_f & \psi_t
\end{bmatrix}\T\,,
\end{equation*}
%
where $\theta$ is the roll angle, $\phi$ is yaw, $\psi_w$ wheel angle,
$\psi_f$ pitch angle and $\psi_t$ the disc angle. The \emph{reward}
$R$ and \emph{loss} $L$ of being in state $\vec x$ is
\[
R(\vec x)\;=\;\exp\big(-\frac{a}{2}d(\vec x)^2-\frac{\phi^2}{2(4\pi)^2}\big),
\qquad\text{and}\qquad L(\vec x)\;=\;1-R(\vec x),
\]
where $a$ is a (positive) constant, and $d^2$ is the the squared
distance between the tip of the unicycle and its ideal location when
upright in the centre of the coordinate system.
\begin{equation*}
  \begin{aligned}
d(\vec x)^2&= d_x ^2 + d_y ^2 + d_z ^2 \qquad \text{with}\\
d_x&= x_c -r \sin \psi_f\\
d_y&= y_c -(r+r_w)\sin\theta\\
d_z&=\big(r_w+r-r_w\cos\theta-r\cos\theta\cos\psi_f\big)^2\\
&=\big(r_w+r-r_w\cos\theta-\tfrac{r}{2}\cos(\theta-\psi_f)
-\tfrac{r}{2}\cos(\theta+\psi_f)\big)^2    
  \end{aligned}
\end{equation*}
Here, $r_w$ is the radius of the wheel and $r$ is the length of the
fork; $x_c$ and $y_c$ are coordinates of the laboratory origin in the
self-centred coordinate system. Apart from those two latter
coordinates, the relevant parts of the state $\vec x$ are the sine and
cosine of $\theta$, the sine of $\psi_f$, and the difference/sum of
the angles $\theta$ and $\psi_f$. Therefore, we define $\alpha :=
\theta-\psi_f$ and $\beta := \theta + \psi_f$. We assume that the
joint distribution of these three variables is approximately Gaussian
\begin{equation*}
\vec z = 
\begin{bmatrix}
x_c \\
\sin\psi_f\\
\hline
y_c \\
\sin\theta\\
\hline
\cos\theta\\
\cos\alpha\\
\cos\beta\\
\hline
\phi
\end{bmatrix}
\quad
\text{and}
\quad
{\cal N}(\bfm, S)\,.
\end{equation*}
The \emph{expected reward} is
\[
\E[R]\;=\;\langle R(\vec x)\rangle_\bfz\;=\;\big\langle
\exp\big(-\tfrac{1}{2}(\bfz-\bfz_0)^\top
T^{-1}(\bfz-\bfz_0)\big)\big\rangle_\bfz \,,
\]
where we defined
\begin{equation*}
\vec z_0 := 
\begin{bmatrix}
0 & 0 & 0 & 0 & 1 & 1 & 1 & 0
\end{bmatrix}\T
\end{equation*}
and
\begin{equation*}
\begin{aligned}
\mat T\inv &=
a\operatorname{diag}\left\{
  \begin{bmatrix}
    1 & -r \\ -r & r^2
  \end{bmatrix}
\;,\;
\begin{bmatrix}
  1 & -(r+r_w) \\ -(r+r_w) & (r+r_w)^2
\end{bmatrix}
\;,\;
\begin{bmatrix}
r_w^2 & \frac{r_wr}{2} & \frac{r_wr}{2}\\
\frac{r_wr}{2} & \frac{r^2}{4} &  \frac{r^2}{4}\\
 \frac{r_wr}{2} & \frac{r^2}{4} &  \frac{r^2}{4}
\end{bmatrix}
\;,\;
\begin{bmatrix}
\frac{1}{16\pi^2}
\end{bmatrix}
\right\}\\
&= a \operatorname{diag}(\mat C_x \mat C_x \T \;,\; \mat C_y \mat C_y \T \;,\; \mat
C_z \mat C_z \T )
\end{aligned}
\end{equation*}
with
\begin{equation*}
\mat C_x =
\begin{bmatrix}
  1 \\ -r  
\end{bmatrix}
\qquad
\mat C_y = 
\begin{bmatrix}
1 \\ - (r + r_w)  
\end{bmatrix}
\qquad\mat C_z = 
\begin{bmatrix}
r_w \\ r/2 \\ r/2
\end{bmatrix}
\end{equation*}


\end{document} 
