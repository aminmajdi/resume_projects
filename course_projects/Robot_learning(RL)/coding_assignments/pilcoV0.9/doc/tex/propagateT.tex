
% This LaTeX was auto-generated from an M-file by MATLAB.
% To make changes, update the M-file and republish this document.



    
    

\subsection*{propagateT.m} 

\begin{par}
\textbf{Summary:} Test derivatives of the propagate function, which computes the mean and the variance of the successor state distribution, assuming that the current state is Gaussian distributed with mean m and covariance matrix s.
\end{par} \vspace{1em}
\begin{verbatim}[dd dy dh] = propagateT(deriv, plant, dynmodel, policy, m, s, delta)\end{verbatim}
\begin{par}
\textbf{Input arguments:}
\end{par} \vspace{1em}
\begin{verbatim}deriv    desired derivative. options:
     (i)    'dMdm' - derivative of the mean of the predicted state
             wrt the mean of the input distribution
     (ii)   'dMds' - derivative of the mean of the predicted state
             wrt the variance of the input distribution
     (iii)  'dMdp' - derivative of the mean of the predicted state
             wrt the controller parameters
     (iv)   'dSdm' - derivative of the variance of the predicted state
             wrt the mean of the input distribution
     (v)    'dSds' - derivative of the variance of the predicted state
             wrt the variance of the input distribution
     (vi)   'dSdp' - derivative of the variance of the predicted state
             wrt the controller parameters
     (vii)  'dCdm' - derivative of the inv(s)*(covariance of the input and
             the predicted state) wrt the mean of the input distribution
     (viii) 'dCds' - derivative of the inv(s)*(covariance of the input and
             the predicted state) wrt the variance of the input distribution
     (ix)   'dCdp' - derivative of the inv(s)*(covariance of the input and
             the predicted state) wrt the controller parameters
plant      plant structure
  .poli    state indices: policy inputs
  .dyno    state indices: predicted variables
  .dyni    state indices: inputs to ODE solver
  .difi    state indices that are learned via differences
  .angi    state indices: angles
  .poli    state indices: policy inputs
  .prop    function handle to function responsible for state
           propagation. Here: @propagated
dynmodel   GP dynamics model (structure)
  .hyp     log-hyper parameters
  .inputs  training inputs
  .targets training targets
policy     policy structure
  .maxU    maximum amplitude of control
  .fcn     function handle to policy
  .p       struct of policy parameters
  .p.\ensuremath{<}\ensuremath{>}    policy-specific parameters are stored here
m          mean of the input distribution
s          covariance of the input distribution
delta      (optional) finite difference parameter. Default: 1e-4\end{verbatim}
\begin{par}
\textbf{Output arguments:}
\end{par} \vspace{1em}
\begin{verbatim}dd         relative error of analytical vs. finite difference gradient
dy         analytical gradient
dh         finite difference gradient\end{verbatim}
\begin{par}
Copyright (C) 2008-2013 by Marc Deisenroth, Andrew McHutchon, Joe Hall, and Carl Edward Rasmussen.
\end{par} \vspace{1em}
\begin{par}
Last modified: 2013-03-21
\end{par} \vspace{1em}

\begin{lstlisting}
function [dd dy dh] = propagateT(deriv, plant, dynmodel, policy, m, s, delta)
\end{lstlisting}


\subsection*{Code} 


\begin{lstlisting}
if nargin < 2,
  randn('seed',24)
  nn = 10;
  E = 4; F = 3;
  m = randn(D,1);
  s = randn(D); s = s*s';

  % Plant ----------------------------------------------------------------
  plant.poli = 1:E;
  plant.dyno = 1:E;
  plant.dyni = 1:E;
  plant.difi = 1:E;
  plant.angi = [];
  plant.prop = @propagated;

  % Policy ---------------------------------------------------------------
  policy.p.w = randn(F,E);
  policy.p.b = randn(F,1);
  policy.fcn = @(policy,m,s)conCat(@conlin,@gSat,policy,m,s);
  policy.maxU = 20*ones(1,F);

  % Dynamics -------------------------------------------------------------
  dynmodel.hyp = zeros(1,E);
  dynmodel.inputs  = randn(nn,E+F);
  dynmodel.targets = randn(nn,E);

end

if nargin < 7; delta = 1e-4; end


D = length(m);

switch deriv

  case 'dMdm'
      [dd dy dh] = checkgrad(@propagateT0, m, delta, plant, dynmodel, policy, s);

  case 'dSdm'
      [dd dy dh] = checkgrad(@propagateT1, m, delta, plant, dynmodel, policy, s);

  case 'dMds'
      [dd dy dh] = checkgrad(@propagateT2, s(tril(ones(D))==1), delta, plant, ...
                                                        dynmodel, policy, m);

  case 'dSds'
      [dd dy dh] = checkgrad(@propagateT3, s(tril(ones(D))==1), delta, plant, ...
                                                        dynmodel, policy, m);

  case 'dMdp'
      p = unwrap(policy.p);
      [dd dy dh] = checkgrad(@propagateT4, p, delta, plant, dynmodel, policy, m, s);

  case 'dSdp'
      p = unwrap(policy.p);
      [dd dy dh] = checkgrad(@propagateT5, p, delta, plant, dynmodel, policy, m, s);
end
\end{lstlisting}

\begin{lstlisting}
function [f, df] = propagateT0(m, plant, dynmodel, policy, s)       % dMdm
if nargout == 1
  M = plant.prop(m, s, plant, dynmodel, policy);
else
  [M, S, dMdm] = plant.prop(m, s, plant, dynmodel, policy);
  df = permute(dMdm,[1,3,2]);
end
f = M;

function [f, df] = propagateT1(m, plant, dynmodel, policy, s)       % dSdm
if nargout == 1
  [M, S] = plant.prop(m, s, plant, dynmodel, policy);
else
  [M, S, dMdm, dSdm] = plant.prop(m, s, plant, dynmodel, policy);
  dd = length(M); df = reshape(dSdm,dd,dd,[]);
end
f = S;

function [f, df] = propagateT2(s, plant, dynmodel, policy, m)       % dMds
d = length(m);
ss(tril(ones(d))==1) = s; ss = reshape(ss,d,d); ss = ss + ss' - diag(diag(ss));
if nargout == 1
  M = plant.prop(m, ss, plant, dynmodel, policy);
else
  [M, S, dMdm, dSdm, dMds] = plant.prop(m, ss, plant, dynmodel, policy);
  dd = length(M); dMds = reshape(dMds,dd,d,d); df = zeros(dd,1,d*(d+1)/2);
  for i=1:dd;
        dMdsi(:,:) = dMds(i,:,:); dMdsi = dMdsi + dMdsi'-diag(diag(dMdsi));
        df(i,:) = dMdsi(tril(ones(d))==1);
  end
end
f = M;

function [f, df] = propagateT3(s, plant, dynmodel, policy, m)       % dSds
d = length(m);
ss(tril(ones(d))==1) = s; ss = reshape(ss,d,d); ss = ss + ss' - diag(diag(ss));
if nargout == 1
  [M, S] = plant.prop(m, ss, plant, dynmodel, policy);
else
  [M, S, dMdm, dSdm, dMds, dSds] = ...
                               plant.prop(m, ss, plant, dynmodel, policy);
  dd = length(M); dSds = reshape(dSds,dd,dd,d,d); df = zeros(dd,dd,d*(d+1)/2);
    for i=1:dd; for j=1:dd
        dSdsi(:,:) = dSds(i,j,:,:); dSdsi = dSdsi+dSdsi'-diag(diag(dSdsi));
        df(i,j,:) = dSdsi(tril(ones(d))==1);
    end; end
end
f = S;

function [f, df] = propagateT4(p, plant, dynmodel, policy, m, s)    % dMdp
policy.p = rewrap(policy.p, p);
if nargout == 1
  M = plant.prop(m, s, plant, dynmodel, policy);
else
  [M, S, dMdm, dSdm, dMds, dSds, dMdp] = plant.prop(m, s, plant, dynmodel, policy);
  df = dMdp;
end
f = M;

function [f, df] = propagateT5(p, plant, dynmodel, policy, m, s)    % dSdp
policy.p = rewrap(policy.p, p);
if nargout == 1
  [M, S] = plant.prop(m, s, plant, dynmodel, policy);
else
  [M, S, dMdm, dSdm, dMds, dSds, dMdp, dSdp] = ...
                                plant.prop(m, s, plant, dynmodel, policy);
  df = dSdp;
end
f = S;
\end{lstlisting}
