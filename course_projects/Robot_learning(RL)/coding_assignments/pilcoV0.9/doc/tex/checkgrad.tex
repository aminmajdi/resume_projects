
% This LaTeX was auto-generated from an M-file by MATLAB.
% To make changes, update the M-file and republish this document.



    
    

\subsection*{checkgrad.m} 

\begin{par}
\textbf{Summary:} checkgrad checks the derivatives in a function, by comparing them to finite differences approximations. The partial derivatives and the approximation are printed and the norm of the difference divided by the norm of the sum is returned as an indication of accuracy.
\end{par} \vspace{1em}

\begin{verbatim}  function [d dy dh] = checkgrad(f, X, e, varargin)\end{verbatim}
    \begin{par}
\textbf{Input arguments:}
\end{par} \vspace{1em}

\begin{lstlisting}
%		f          function handle to function that needs to be checked.
%              The function f should be of the type [fX, dfX] = f(X, varargin)
%              where fX is the function value and dfX is the gradient of fX
%              with respect to the parameters X
%   X          parameters (can be a vector or a struct)
%   e          small perturbation used for finite differences (1e-4 is good)
%   varargin   other arguments that are passed on to the function f
%
%
% *Output arguments:*
%
%   d          relative error of analytical vs. finite difference gradient
%   dy         analytical gradient
%   dh         finite difference gradient
%
%
% Copyright (C) 2008-2013 by
% Marc Deisenroth, Andrew McHutchon, Joe Hall, and Carl Edward Rasmussen.
%
% Last modified: 2013-03-21
%
\end{lstlisting}


\subsection*{High-Level Steps} 

\begin{enumerate}
\setlength{\itemsep}{-1ex}
   \item Analytical gradient
   \item Numerical gradient via finite differences
   \item Relative error
\end{enumerate}

\begin{lstlisting}
function [d dy dh] = checkgrad(f, X, e, varargin)
\end{lstlisting}


\subsection*{Code} 


\begin{lstlisting}
% 1. Analytical gradient
Z = unwrap(X); NZ = length(Z);                 % number of input variables
[y dy] = feval(f, X, varargin{:});             % get the partial derivatives dy
[D E] = size(y); y = y(:); Ny = length(y);     % number of output variables
if iscell(dy) || isstruct(dy); dy = unwrap(dy); end;
dy = reshape(dy,Ny,NZ);

% 2. Finite difference approximation
dh = zeros(Ny,NZ);
for j = 1:NZ
  dx = zeros(length(Z),1);
  dx(j) = dx(j) + e;                               % perturb a single dimension
  y2 = feval(f, rewrap(X,Z+dx), varargin{:});
  y1 = feval(f, rewrap(X,Z-dx), varargin{:});
  dh(:,j) = (y2(:) - y1(:))/(2*e);
end

% 3. Compute error
% norm of diff divided by norm of sum
d = sqrt(sum((dh-dy).^2,2)./sum((dh+dy).^2,2));
small = max(abs([dy dh]),[],2) < 1e-5; % small derivatives are poorly tested ...
d(d > 1e-3 & small) = NaN;             % ... by finite differences
d = reshape(d,D,E);

disp('   Analytic  Numerical');
for i=1:Ny;
    disp([dy(i,:)' dh(i,:)']);                           % print the two vectors
    fprintf('d = %e\n\n',d(i))
end

if Ny > 1; disp('For all outputs, d = '); disp(d); end; fprintf('\n');
\end{lstlisting}
