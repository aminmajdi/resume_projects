
% This LaTeX was auto-generated from an M-file by MATLAB.
% To make changes, update the M-file and republish this document.



    
    

\subsection*{gpr.m} 

\begin{par}
\textbf{Summary:} Gaussian process regression, with a named covariance function. Two modes are possible: training and prediction: if no test data are given, the function returns minus the log likelihood and its partial derivatives with respect to the hyperparameters; this mode is used to fit the hyperparameters. If test data are given, then (marginal) Gaussian predictions are computed, whose mean and variance are returned. Note that in cases where the covariance function has noise contributions, the variance returned in S2 is for noisy test targets; if you want the variance of the noise-free latent function, you must substract the noise variance.
\end{par} \vspace{1em}
\begin{par}
usage: [nlml dnlml] = gpr(logtheta, covfunc, x, y)    or: [mu S2]  = gpr(logtheta, covfunc, x, y, xstar)
\end{par} \vspace{1em}
\begin{par}
where:
\end{par} \vspace{1em}
\begin{verbatim}logtheta is a (column) vector of log hyperparameters
covfunc  is the covariance function
x        is a n by D matrix of training inputs
y        is a (column) vector (of size n) of targets
xstar    is a nn by D matrix of test inputs
nlml     is the returned value of the negative log marginal likelihood
dnlml    is a (column) vector of partial derivatives of the negative
              log marginal likelihood wrt each log hyperparameter
mu       is a (column) vector (of size nn) of prediced means
S2       is a (column) vector (of size nn) of predicted variances\end{verbatim}
\begin{par}
For more help on covariance functions, see "help covFunctions".
\end{par} \vspace{1em}
\begin{par}
(C) Copyright 2006 by Carl Edward Rasmussen (2006-03-20).
\end{par} \vspace{1em}

\begin{lstlisting}
function [out1, out2] = gpr(logtheta, covfunc, x, y, xstar)
\end{lstlisting}


\subsection*{Code} 


\begin{lstlisting}
if ischar(covfunc), covfunc = cellstr(covfunc); end % convert to cell if needed
[n, D] = size(x);
if eval(feval(covfunc{:})) ~= size(logtheta, 1)
  error('Error: Number of parameters do not agree with covariance function')
end

K = feval(covfunc{:}, logtheta, x);    % compute training set covariance matrix

L = chol(K)';                        % cholesky factorization of the covariance
alpha = solve_chol(L',y);

if nargin == 4 % if no test cases, compute the negative log marginal likelihood

  out1 = 0.5*y'*alpha + sum(log(diag(L))) + 0.5*n*log(2*pi);

  if nargout == 2               % ... and if requested, its partial derivatives
    out2 = zeros(size(logtheta));       % set the size of the derivative vector
    W = L'\(L\eye(n))-alpha*alpha';                % precompute for convenience
    for i = 1:length(out2)
      out2(i) = sum(sum(W.*feval(covfunc{:}, logtheta, x, i)))/2;
    end
  end

else                    % ... otherwise compute (marginal) test predictions ...

  [Kss, Kstar] = feval(covfunc{:}, logtheta, x, xstar);     %  test covariances

  out1 = Kstar' * alpha;                                      % predicted means

  if nargout == 2
    v = L\Kstar;
    out2 = Kss - sum(v.*v)';
  end

end
\end{lstlisting}
